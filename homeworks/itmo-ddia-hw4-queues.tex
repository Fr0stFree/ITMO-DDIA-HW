\documentclass[12pt]{article}
\usepackage[dvipsnames]{xcolor}
\usepackage{minted}
% sudo tlmgr install minted
\usepackage{hyperref}
\usepackage[margin=1in]{geometry} 
\usepackage{amsmath,amsthm,amssymb}
\usepackage{graphicx}
\usepackage[utf8]{inputenc}
\usepackage[T2A]{fontenc}
\usepackage[russian,english]{babel}

\hypersetup{
    colorlinks=true,
    linkcolor=blue,
    filecolor=magenta,      
    urlcolor=blue,
    pdfpagemode=FullScreen,
}

\newenvironment{problem}{
    \par\textbf{Задание: }\ignorespaces
}

\newenvironment{solution}
{
  \par\textbf{Решение:}\par\bigbreak
  \begingroup
}
{
  \par\bigbreak\hfill$\square$%
  \par\endgroup
}

\begin{document}
 
\title{
    \large Принципы построения высоконагруженных систем \\
    \normalsize Институт прикладных компьютерных наук ИТМО
    \bigbreak 
    \LARGE
    Домашнее задание 4. Проектирование высоконагруженной системы \\ 
    \bigbreak \normalsize
    Георгий Семенов \\
    \texttt{georgii.v.semenov@mail.ru} \\
   \color{red}Мягкий дедлайн: \color{red}{Вт, 23.12.2025, 23:59 МСК} \\
    Жесткий дедлайн: {Вт, 30.12.2025, 23:59 МСК}
}
\author{
    \color{red}{Имя Фамилия} \\
    \normalsize
    DDIA25-HW4-{\color{red}NameSurname}.pdf
}

\maketitle

\noindent\rule{\textwidth}{1pt} \bigbreak

Это домашнее задание – финальное в курсе и представляет собой курсовую работу, которую необходимо очно защитить.
В ней необходимо спроектировать высоконагруженную систему, рассматривая те аспекты, которые были
освещены в рамках занятий и домашних заданий. Давайте их структурированно перечислим:

\begin{itemize}
	\item \textbf{SLA} (uptime, latency) – то, что система предоставляет пользователям;
	\item \textbf{Пропускная способность} системы ограничивается её bottlenecks – аппаратными (CPU, RAM, Disk), сетевыми (bandwidth), слоями хранения данных (СУБД, кэш и т.д.) и пр.;
	\item Мы хотим оставить себе возможность увеличивать пропускную способность системы с помощью \textbf{масштабирования} (вертикального и горизонтального);
	\item Свойство возможности наблюдения за ходом работы системы – \textbf{observability} – в real-time обеспечивается мониторингом SLO/SLI, проверками здоровья;
	\item \textbf{Устойчивость} системы к отказам (resilience) обеспечивается избеганием \textbf{SPoF} и стратегиями устойчивости (retry, failover, bulkhead, hedging, circuit breaker и пр.);
	\item В случае отказов (внешней перегрузки из-за DDoS, и из-за внутренних) хочется аккуратно сдеградировать – \textbf{graceful degradation}, \textbf{failover};
	\item \textbf{Балансировка нагрузки} между компонентами системы \textbf{в различных зонах} позволяет равномерно распределять нагрузку (RR, LC, IP Hash и т.д.) и избегать перегрузок отдельных компонентов;
	\item Стратегия управления инцидентами (\textbf{incident management}) и их ретроспективный анализ (\textbf{post mortem}) помогают повышать \textbf{надежность} системы со временем;
	\item \textbf{Слой хранения данных (stateful)} реализуется на основе хранилищ (ACID/BASE), кэшей (стратегии кэширования); в паттернах \textbf{федерации} и \textbf{шардирования};
	\item На практике мы вынуждены балансировать между \textbf{latency} и \textbf{consistency} системы (PACELC-теорема);
	\item Важный паттерн проектирования высоконагруженных систем - \textbf{очереди} (event sourcing, microservices, CQRS)
\end{itemize}

Возьмите любой свой проект (в т.ч. можно прошлый учебный) и спроектируйте его высоконагруженную архитектуру, освещая \textbf{все аспекты, упомянутые выше,} и следуя схеме,
предложенной ниже.

\break

\section{Введение}{}

\subsection{Предметная область}

\textit{что за предметную область вы рассматриваете, проектируя вашу систему? какие у нее особенности? какая проблема решается?}

\subsection{Требования к системе}

\subsubsection{Функциональные требования}

\textit{что должна делать система? какие пользовательские истории она должна поддерживать?}

\subsubsection{Нефункциональные требования}

\textit{SLA (uptime, latency), SLO/SLI (error rate и т.д.), throughput (RPS), scalability (какие бизнес-риски и насколько заставляют масштабировать систему в будущем?). Обоснуйте выбор метрик}

\section{Архитектура системы}{}

\subsection{Потоки данных}

\textit{deployment-диаграмма системы; DFD-диаграмма системы; слой хранения данных (хранилища, кэши), федерация и шардирование; как реализуется микросервисная архитектура, какие очереди (event sourcing) используются? какая модель репликации, consistency?}

\subsection{Отказоустойчивость}

\textit{балансировка, как обеспечивается graceful degradation, zone redundancy, failover, resilience strategies?}

\textit{рефлексия: почему нет единственных точек отказа (или все же есть)? какой фактор вмешательства человека? MTTR? какая целевая архитектура может быть предложена, если приведенная выше - текущая?}

\section{Эксплуатация и управление надежностью}{}

\subsection{Ресурсы и их масштабирование}

\textit{Предложите количество ресурсов CPU, RAM, LAN, NVMe для каждого пода микросервиса в вашей архитектуре; количество подов; пропускную способность пода; как обеспечивается автоматическое горизонтальное масштабирование под нагрузкой? }

\subsection{Мониторинг и дежурная смена}

\textit{Отслеживаемые SLO/SLI, как реализуются столпы observability, организационное устройство дежурной смены}

\subsection{Реагирование на инциденты}

\textit{Алерты, runbooks, порядок эскалации инцидентов, коммуникация в команде и с пользователями
; disaster recovery (backup-стратегия, recovery процедуры, RTO/RPO)}

\subsection{Порядок расследования инцидентов и проведение постмортемов}

\textit{Методы расследования инцидентов, протокол проведения встречи с разборами, артефакты разбора, отслеживание выполнения артефактов}

\section{Заключение}{}

\textit{Какая работа была проделана, рефлексия, какие направления дальнейшего проектирования SRE-процессов для системы?}
 
\end{document}
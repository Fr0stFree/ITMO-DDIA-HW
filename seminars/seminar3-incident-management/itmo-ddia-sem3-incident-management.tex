\documentclass{beamer}
\usetheme{Madrid}
\usepackage{minted}
% sudo tlmgr install minted
\usepackage[utf8]{inputenc}
\usepackage[T2A]{fontenc}
\usepackage[russian,english]{babel}
\usepackage{tikz}
\usetikzlibrary{arrows.meta, positioning}

% Footline: show current section (left) and frame numbers (right)
\setbeamertemplate{footline}{%
  \leavevmode%
  \hbox{%
    \begin{beamercolorbox}[wd=.78\paperwidth,ht=2.5ex,dp=1ex,left]{author in head/foot}%
      \hspace*{1em}\usebeamerfont{footline}\insertsectionhead%
    \end{beamercolorbox}%
    \begin{beamercolorbox}[wd=.22\paperwidth,ht=2.5ex,dp=1ex,right]{date in head/foot}%
      \usebeamerfont{footline}\insertframenumber{} / \inserttotalframenumber\hspace*{1em}%
    \end{beamercolorbox}%
  }%
}

\newcommand{\parpause}[1]{\only<+->{#1\par}}

\AtBeginSection[]
{
  \begin{frame}
    \frametitle{Содержание}
    \tableofcontents[currentsection]
  \end{frame}
}
\title{Семинар 3. Устойчивость и управление инцидентами}
\subtitle{Принципы построения высоконагруженных систем}
\author{Георгий Семенов}
\institute{Институт прикладных компьютерных наук \\ Университет ИТМО}
\date{осень 2025}

\begin{document}

\frame{\titlepage}

\section{Введение}

\begin{frame}
  \frametitle{Оставшиеся активности и домашние задания}
  \begin{itemize}
    \item Лекция + Семинар 4: Очереди, собьтия и т.п.
    \item Домашнее задание 3: запрограммировать стратегии устойчивости
    \item Домашнее задание 4: небольшая курсовая работа с устной защитой
  \end{itemize}
\end{frame}

\begin{frame}
  \frametitle{Вспоминаем лекцию}
  \begin{itemize}
    \item SRE: SLA/SLO/SLI; расчет совокупного SLA
    \item Active-passive redundancy (failover): cold, warm, hot
    \item Active-active redundancy: load balancing, capacity planning
    \item Geographic redundancy
    \item Liveness probe, readiness probe
    \item Graceful degradation
    \item Circuit breaker
    \item Retry patterns (идемпотентность): retry, exponential backoff, jitter
    \item Bulkhead (изоляция ресурсов)
    \item Rate limiting (+ backpressure) - leaky bucket, token bucket, fixed/sliding window
  \end{itemize}
\end{frame}

\section{Стратегии устойчивости (resilience)}

\begin{frame}
  \frametitle{Каскадные сбои}

  Каскадные отказы – те, что распространяются с течением времени в результате
  положительной обратной связи\footnote{SRE Book. Глава 22 - Справляемся с каскадными сбоями}.

  \begin{itemize}
    \item Каскадный – т.е. прорывающийся вглубь через границы между компонентами системы
    \item Перегруженность сервера: приняли, подупали, поретраили, и сами себя заDDoS-или
    \item Истощение CPU: переполнение очередей, зависание потоков, троттлинг (\texttt{>100\%})
    \item Истощение RAM: падающие поды, <<спираль смерти GC>>, снижение частоты попадания кэша
    \item и т.д.
  \end{itemize}
\end{frame}

\begin{frame}
  \frametitle{Нейрологическая метафора}

  Серотониновый шторм в синаптических щелях между нейронами:
  \begin{itemize}
    \item Должен быть <<тонус>>
    \item Переполнение приводит к плавной деградации (или нет – тогда шторм) 
  \end{itemize}

  \begin{center}
    \includegraphics[width=0.6\linewidth,keepaspectratio]{images/synaptic.png}
  \end{center}
\end{frame}

\begin{frame}
  \frametitle{Как предотвращать каскадные сбои?}

  \begin{itemize}
    \item Предварительно нагрузочно тестировать
    \item Graceful degradation: уметь отправлять деградированные результаты
    \item Rate limiting + backpressure
    \item Планирование производительности
    \item Управление очередями (thread pool -> concurrency, aka go, userver и т.п.)
  \end{itemize}
\end{frame}

\begin{frame}
  \frametitle{Как мгновенно реагировать на каскадные сбои?}

  \begin{itemize}
    \item Увеличить количество ресурсов
    \item Прекратить выполнять проверки на сбои/<<гибель>>
    \item Перезапускать серверы
    \item Отбрасывать трафик (в т.ч. <<плохой>>)
    \item Автоматические/ручные деградации
  \end{itemize}
\end{frame}

\begin{frame}
  \frametitle{Проблема холодного старта}

  \begin{itemize}
    \item Холодный кэш
    \item Холодный JIT (например, для Java)
  \end{itemize}
\end{frame}

\begin{frame}
  \frametitle{Retries: модель ответов в HTTP}

  \begin{itemize}
    \item \textbf{2xx} – успешный ответ с полезной нагрузкой в теле ответа.
    \item \textbf{4xx} – ошибка со стороны клиента (некорректный запрос, ошибка аутентификации или авторизации и т.п.).
    \item \textbf{5xx} – внутренняя ошибка сервера.
    \item \textbf{Timeout} – клиентский таймаут ожидания ответа от сервера на текущий запрос.
  \end{itemize}
\end{frame}

\begin{frame}
  \frametitle{Retries: типы бюджетов}

  \begin{itemize}
    \item \textbf{Retry budget} – максимальное количество подзапросов, которым мы можем нагрузить сервер за один запрос, в частности:
    \begin{itemize}
      \item \textbf{Fast errors budget} – <<быстрые>> ошибки на подзапросах, т.е. с быстрым временем отказа (обычно \texttt{4xx}).
      \item \textbf{Failures budget} – <<тяжелые>> ошибки, т.е. с некоторым ожидаемым временем работы до падения подзапроса (обычно \texttt{5xx}).
      \item \textbf{Timeout budget} – ошибки типа <<timeout>>, т.е. когда подзапрос завершается из-за истечения клиентского таймаута.
    \end{itemize}
    \item \textbf{Latency budget} – общее время ожидания ответа от сервера на все попытки выполнения запроса.
    \begin{itemize}
      \item в частности – \textbf{Subrequest latency budget} – время ожидания одного подзапроса.
    \end{itemize}
  \end{itemize}
\end{frame}

\begin{frame}
  \frametitle{Немного про ТМО}

  \begin{itemize}
    \item Теория массового обслуживания - раздел математики
    \item Центральный пример - распределение Пуассона (в ДЦ каждый день меняют по три диска)
  \end{itemize}
\end{frame}

\begin{frame}
  \frametitle{Хеджирование}
  \href{https://grpc.io/docs/guides/request-hedging/}{Хеджирование} – спекулятивная повторная отправка дополнительных запросов с целью
  минимизировать время ответа на запрос.

  \begin{itemize}
    \item Первый подзапрос должен отправляться на первый по порядку \textit{интерфейс отправки подзапроса} из их списка.
    \item Если в течение параметризуемого времени \texttt{hedging\_delay} не приходит ответ на первый подзапрос, 
    следующие подзапросы должны параллельно отправляться на оставшиеся \textit{интерфейсы отправки подзапроса}.
    \item Стратегия завершается успешно, когда приходит первый успешный ответ от любого из подзапросов.
    \item Стратегия завершается ошибкой, когда все попытки завершились или исчерпан бюджет на latency запроса.
  \end{itemize}
\end{frame}

\begin{frame}
  \frametitle{Хеджирование: case 1}
  
   \begin{center}
    \includegraphics[width=0.9\linewidth,keepaspectratio]{images/hedging1.png}
  \end{center}

\end{frame}

\begin{frame}
  \frametitle{Хеджирование: case 2}
  
   \begin{center}
    \includegraphics[width=0.9\linewidth,keepaspectratio]{images/hedging2.png}
  \end{center}

\end{frame}

\section{Управление инцидентами}


  \begin{frame}
    \frametitle{Инцидент - спекулятивное определение}

    \begin{itemize}
      \item В 3 ночи поднимаются лиды и разработчики
      \item Дружно каскадно перезапускают упавшие сервисы, пока утром не польется активный трафик
      \item Сидят в зуме и интересуются, как кого что импактит внутри
    \end{itemize}
  \end{frame}

  \begin{frame}
    \frametitle{Кто такая дежурная смена?}

    \begin{itemize}
      \item Существуют линии поддержки - Support Lines:
      \item SL1 - поддержка пользователей (принимает сообщения об отказах)
      \item SL2 - дежурная смена (24 на 7) + алертинг + мониторинг (подтверждает идентификацию и эскалацию инцидентов)
      \item SL3 - ответственные команды и разработчики
    \end{itemize}
  \end{frame}

  \begin{frame}
    \frametitle{Эскалация - глоссарий}

    \begin{itemize}
      \item Вызванивать
      \item Будить
      \item Подключаться
      \item Влияние
      \item Подскажите, какой сейчас статус
    \end{itemize}
  \end{frame}
 
  \begin{frame}
    \frametitle{Авиа- метафора}

    Человек должен <<успевать>> за системой:
    \begin{itemize}
      \item Интерпретируемость автоматического реагирования системы на отказы
      \item Адекватность архитектуры и ее мониторинга
      \item Искусство координации инцидентным звонком
    \end{itemize}

    \begin{center}
      \includegraphics[width=0.75\linewidth,keepaspectratio]{images/emergency.png}
    \end{center}
  \end{frame}

  \begin{frame}
    \frametitle{Post mortem}

    \begin{itemize}
      \item Инцидент - это срабатывание риска, связанное с упущенной прибылью
      \item Цель реагирования – как можно скорее \textbf{устранить влияние}
      \item Цель разбора – \textbf{установить причину} инцидента, чтобы устранить ее
      \item Разбор не должен обвинять никого, он должен быть нейтральный
    \end{itemize}

  \end{frame}

    \begin{frame}
    \frametitle{Как разбирать инцидент?}

    Проблемные вопросы:
    \begin{itemize}
      \item Что случилось? (и как долго?)
      \item Как (кто?) заметил? (благодаря случайности или процессу?)
      \item Какие были warning sings?
      \item Как можно оценить влияние?
      \item Какие action items можно выделить для устранения причин (предотвращения в будущем)?
      \item Что в итоге сделали?
    \end{itemize}

  \end{frame}

\section{Инциденты: case-study}

  \begin{frame}
    \frametitle{IRL: обновление DNS в пятницу вечером}

    \begin{itemize}
      \item Пятница вечер, 22:00 – с подов сервиса пропадает доступ (обнаруживается через
      мониторинг сервиса) к двум из трех балансировщиков СУБД
      \item Трафик на оставшийся кластер возрос в $3$ раза
      \item Выясняется, что в это время обновили конфигурацию DNS (добавили ipv6 AAAA-записи)
      \item Вопрос 1: почему это помешало работе системы?
      \item Вопрос 2: какую можно установить причину для инцидента?
      % нарушение регламента работ
    \end{itemize}

  \end{frame}

  \begin{frame}
    \frametitle{BGP routing issue 2021 (6 часов)}

     \begin{center}
      \includegraphics[width=1.0\linewidth,keepaspectratio]{images/bgp.png}
    \end{center}

  \end{frame}

  \begin{frame}
    \frametitle{AWS S3 Outage 2017 (4 часа)}

    \begin{itemize}
      \item February 28, 2017, from mid-morning to early afternoon PST (around 9:40 AM to 12:36 PM PST)
      \item Location: Primarily the US-East-1 (N. Virginia) region, the primary AWS region.
      \item Cause: An authorized S3 team member executed a command with an incorrect input while trying to debug a slow billing system, inadvertently removing more servers than intended, affecting crucial S3 subsystems.
      \item An authorized S3 team member executed a command with an incorrect input while trying to debug a slow billing system, inadvertently removing more servers than intended, affecting crucial S3 subsystems.
    \end{itemize}

  \end{frame}

  \begin{frame}
    \frametitle{CloudFlare Outage 2019 (27 мин)}

    \begin{center}
      \includegraphics[width=1.0\linewidth,keepaspectratio]{images/cloudflare.png}
    \end{center}

  \end{frame}

\section{Как проектировать высоконагруженные системы?}

  \begin{frame}
    \frametitle{Архитектура}

    Набор компонентов и связей между ними (но не только); набор решений, необходимых для разработки системы:

    \begin{itemize}
      \item Системная архитектура - какие сервисы, как расположены, как общаются
      \item Программная архитектура – парадигма программирования, модульная организация, паттерны и т.п.
      \item Архитектура данных – какие данные, схема хранения, СУБД и т.п.
    \end{itemize}

  \end{frame}

  \begin{frame}
    \frametitle{Диаграмма развертывания}

    Можно нарисовать в Visual Paradigm Online.

    \begin{center}
      \includegraphics[width=0.9\linewidth,keepaspectratio]{images/deployment-diagram.png}
    \end{center}

  \end{frame}

  \begin{frame}
    \frametitle{Dataflow diagram}

    Можно нарисовать в Visual Paradigm Online.

    \begin{center}
      \includegraphics[width=0.9\linewidth,keepaspectratio]{images/dataflow-diagram.png}
    \end{center}

  \end{frame}

  \begin{frame}
    \frametitle{Как пройти процесс проектирования?}

    \begin{itemize}
      \item Итеративный подход – задаем себе вопросы по кругу, пока не успокоимся
      \item Не является ли компонент X единой точкой отказа? Как это исправить? – Исправляем.
      \item Моделируем ситуации отказа сразу же при проектировании – отказ сервиса, сетевой разрыв,
      заполнение СУБД, DDoS и т.п. – моделируем инцидент, исправляем первопричины
    \end{itemize}

  \end{frame}

\section{Итоги}

\begin{frame}
  \frametitle{Итоги}
  \begin{itemize}
    \item Обсудили паттерны устойчивости и каскадные сбои
    \item Погрузились в аспекты IM
    \item Выдано третье + четвертое домашнее задание
  \end{itemize}
\end{frame}

\end{document}